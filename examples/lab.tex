\documentclass{dsavarticle}

\usepackage{dsavlabtitle}

\newcommand{\doctitle}{Методы добычи железной руды на Альфа Центавра}

\hypersetup
{
  pdfauthor={Дмитрий Савченко},
  pdftitle={\doctitle},
  pdfsubject = {\doctitle}
}

\chair{Кафедра технологий программирования}
\author{Савченко\,Д.\,А.}
\group{09-ИТ-2}
\checkedby{Иванов~С.~П.}
\discipline{Языки программирования}
\worktype{Лабораторная работа}
\worknum{1}
\title{\doctitle}
\city{Полоцк}

\begin{document}

\maketitle

\tableofcontents
\newpage


% Stub texts generated by http://vesna.yandex.ru/

\section{Цели занятия}

Если предварительно подвергнуть объекты длительному вакуумированию,
сверхновая возможна. Почвенная влага, как бы это ни казалось парадоксальным,
выбирает дип-скай объект, об интересе Галла к астрономии и затмениям Цицерон
говорит также в трактате "О старости" (De senectute). Растрескивание непрерывно.
Зенитное часовое число, как следует из полевых и лабораторных наблюдений,
одномерно дает водонасыщенный агробиогеоценоз - это солнечное затмение
предсказал ионянам Фалес Милетский.

\section{Описание проделанной работы}

В условиях электромагнитных помех, неизбежных при полевых измерениях,
не всегда можно опредлить, когда именно поверхность раздела фаз притягивает
подбур, и этот процесс может повторяться многократно. С другой стороны,
определение содержания в почве железа по Тамму показало, что лёсс
выслеживает тангенциальный режим, как это случилось в 1994 году с кометой
Шумейкеpов-Леви 9. Фаза традиционно решает шаг смешения без обмена
зарядами или спинами. Силовое поле ищет магнит почти так же, как
в резонаторе газового лазера. Пульсар гомогенно иссушает промывной
часовой угол одинаково по всем направлениям.

\section{Вывод}

Отвесная линия оценивает почвообразовательный процесс, что дает возможность
использования данной методики как универсальной. Потускула неверифицируемо
иллюстрирует Юпитер, но никакие ухищрения экспериментаторов не позволят
наблюдать этот эффект в видимом диапазоне. Силовое поле колеблет вращательный
бозе-конденсат только в отсутствие тепло- и массообмена с окружающей средой.
В лабораторных условиях было установлено, что включение продуцирует глей
одинаково по всем направлениям.

\end{document}
