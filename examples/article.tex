\documentclass{dsavarticle}

\usepackage{dsavcommon}
\usepackage{dsavlistings}
\usepackage{float}

\newcommand{\fs}{\textit}

\newcommand\thetitle{Lorem ipsum}
\newcommand\theauthor{Дмитрий Савченко}

\author{\theauthor}
\title{\thetitle}

\hypersetup
{
  pdfauthor={\theauthor},
  pdfsubject={\thetitle},
  pdftitle={\thetitle}
}

\begin{document}

\maketitle
\newpage

\tableofcontents
\newpage


% Stub texts generated by http://vesna.yandex.ru/

\section*{Введение}
\addcontentsline{toc}{section}{Введение}

Философия эксперментально верифицируема. Расстояния планет от Солнца
возрастают приблизительно в геометрической прогрессии (правило Тициуса — Боде),
где перигей осмысленно расщепляет изобарический взрыв, не учитывая мнения
авторитетов. Позитивизм абстрактен. Изолируя область наблюдения от посторонних
шумов, мы сразу увидим, что поток ускоряет неоднозначный гений в полном
соответствии с законом сохранения энергии. Веданта воспроизводима в
лабораторных условиях.


\section{Изображения}

Рисунок~\ref{img:100-percent} занимает всю ширину страницы, а~рисунок~%
\ref{img:50-percent}~--- половину ширины.

\begin{figure}[H]
  \image{image.pdf}
  \caption{Рисунок шириной в 100~\%}
  \label{img:100-percent}
\end{figure}

\begin{figure}[H]
  \image[0.5]{image.pdf}
  \caption{Рисунок шириной в 50~\%}
  \label{img:50-percent}
\end{figure}


\section{Таблица}

Если для простоты пренебречь потерями на теплопроводность,
то видно, что жидкость параллельна. Приливное трение, несмотря
на внешние воздействия, оценивает объект - все дальнейшее далеко
выходит за рамки текущего исследования и не будет здесь рассматриваться.
Непосредственно из законов сохранения следует, что расслоение представляет
собой квантовый параметр - все дальнейшее далеко выходит за рамки текущего
исследования и не будет здесь рассматриваться. Когда речь идет
о галактиках, звезда ускоряет межпланетный кварк, даже если пока
мы не можем наблюсти это непосредственно. Струя недоступно
отклоняет короткоживущий метеорный дождь, однако большинство
спутников движутся вокруг своих планет в ту же сторону, в какую
вращаются планеты.

\begin{table}[H]
  \caption{Lorem ipsum}
  \label{tbl:lorem-ipsum}

  \begin{tabularx}{\textwidth}{ |P{110pt}|X|c| }
    \hline

    \bfseries
    Lorem &

    \bfseries
    Ipsum &

    \bfseries
    Dolor sit

    \tabularnewline
    \hline

    Lorem ipsum dolor sit Lorem ipsum dolor sit &
    Lorem ipsum dolor sit Lorem ipsum dolor sit &
    Lorem ipsum dolor sit Lorem ipsum dolor sit \\
    \hline

    Lorem ipsum dolor sit Lorem ipsum dolor sit &
    Lorem ipsum dolor sit Lorem ipsum dolor sit &
    Lorem ipsum dolor sit Lorem ipsum dolor sit \\
    \hline
  \end{tabularx}

\end{table}

Ударная волна, несмотря на некоторую вероятность коллапса, точно
отражает случайный перигелий, даже если пока мы не можем наблюсти
это непосредственно. Зенитное часовое число жизненно переворачивает
эксцентриситет, данное соглашение было заключено на 2-й международной
конференции "Земля из космоса - наиболее эффективные решения".
Химическое соединение, в согласии с традиционными представлениями,
перечеркивает циркулирующий фотон, что лишний раз подтверждает
правоту Эйнштейна. Химическое соединение устойчиво поглощает
экваториальный метеорный дождь, данное соглашение было заключено на
2-й международной конференции "Земля из космоса - наиболее
эффективные решения". Вихрь сжимает далекий экситон, таким образом,
атмосферы этих планет плавно переходят в жидкую мантию. Соединение
ищет погранслой, и этот процесс может повторяться многократно.


\section{Листинги}

В~листингах \ref{lst:blowfish-header} и \ref{lst:blowfish-source}
приведены тексты файлов \fs{blowfish.h} и \fs{blowfish.cpp}
соответственно.

\begin{listing}
  \caption{Листинг файла \fs{blowfish.h}}
  \source[cpp]{listings/blowfish.h}
  \label{lst:blowfish-header}
\end{listing}

\begin{listing}
  \caption{Листинг файла \fs{blowfish.cpp}}
  \source[cpp]{listings/blowfish.cpp}
  \label{lst:blowfish-source}
\end{listing}


\section*{Заключение}
\addcontentsline{toc}{section}{Заключение}

Колебание когерентно искажает ускоряющийся математический горизонт,
и это неудивительно, если вспомнить квантовый характер явления.
Различное расположение перманентно поглощает кварк, хотя для имеющих
глаза-телескопы туманность Андромеды показалась бы на небе величиной
с треть ковша Большой Медведицы. Химическое соединение отталкивает
гравитационный газ одинаково по всем направлениям. Эксимер возбуждает
гидродинамический удар, генерируя периодические импульсы синхротронного
излучения.

\nocite{*}
\bibliography{literature}

\appendix

\section{Рисунок}
\label{apdx:image}
\image{image.pdf}

\section{Листинг}
\label{apdx:listing}
\source[cpp]{listings/blowfish.h}

\end{document}
